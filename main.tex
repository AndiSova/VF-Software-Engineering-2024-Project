% This is samplepaper.tex, a sample chapter demonstrating the
% LLNCS macro package for Springer Computer Science proceedings;
% Version 2.21 of 2022/01/12
%
\documentclass[runningheads]{llncs}
%
\usepackage[T1]{fontenc}
% T1 fonts will be used to generate the final print and online PDFs,
% so please use T1 fonts in your manuscript whenever possible.
% Other font encondings may result in incorrect characters.
%
\usepackage{graphicx}
\usepackage[printonlyused,withpage]{acronym}

% Used for displaying a sample figure. If possible, figure files should
% be included in EPS format.
%
% If you use the hyperref package, please uncomment the following two lines
% to display URLs in blue roman font according to Springer's eBook style:
%\usepackage{color}
%\renewcommand\UrlFont{\color{blue}\rmfamily}
%\urlstyle{rm}
%
\begin{document}
%
\title{MiniSat Report}
%
%\titlerunning{Abbreviated paper title}
% If the paper title is too long for the running head, you can set
% an abbreviated paper title here
%
\author{Șova Dumitru Ștefan Andrei \and
Andries Rafael Gabriel \and
Stoentel Alexandru-Eduard \and Nenescu Eugeniu}
%
\authorrunning{F. Author et al.}
% First names are abbreviated in the running head.
% If there are more than two authors, 'et al.' is used.
%
\institute{West University of Timișoara, Bulevardul Vasile Pârvan 4, Timișoara 300223}
%
\maketitle              % typeset the header of the contribution
%
\begin{abstract}

Since time immemorial, logic has helped us make sense of the world that surrounds us, from our ancestors' philosophical inquiries to its use in modern mathematics and informatics. Logic bridges the gap between formal and informal knowledge, proving essential in fields w diverse as databases, programming languages (e.g., Prolog), and propositional logic. This paper focuses on a compelling area within informatics: the \ac{SAT}, a foundational challenge in computational complexity. SAT asks whether a set of variable assignments can satisfy a Boolean formula and stands as the first problem proven to be NP-complete. Our main focus will be on MiniSat, an SAT solver designed to address this challenge efficiently. For the first time, such solvers enabled tackling practical problems with millions of variables, benefiting both research and industry.

This report includes benchmarks from recent SAT competitions, alongside minor modifications to MiniSat’s original code, to compare results and evaluate its performance across different problem instances. With timing data and outputs available on GitHub, this analysis highlights MiniSat’s capabilities and underscores the broader impact of SAT solvers in advancing computational problem-solving. The following link holds the contents of the project: \url{https://github.com/AndiSova/VF-Software-Engineering-2024-Project}

\keywords{MiniSat  \and SAT \and logic.}
\end{abstract}
%
%
%
\section{Introduction}

The SAT (Satisfiability) problem, central in computer science and logic, has gained significant traction in recent years as SAT solvers demonstrate their utility across a growing range of applications. From electronic design automation (EDA) to artificial intelligence and optimization, SAT solvers are increasingly leveraged to tackle complex, real-world challenges. Their success is amplified by the ability to encode various problem types into SAT formulations efficiently, allowing for high-speed solutions that can handle industry-specific requirements.

Despite their success, developing or even modifying an SAT solver presents a substantial challenge. Modern SAT solvers incorporate intricate algorithms and optimizations, such as conflict-driven clause learning (CDCL) and non-chronological backtracking, which require expertise in both theory and implementation to be utilized effectively. Although foundational SAT techniques are well-documented, the specific implementation details that enhance solver performance and adaptability often remain obscure or inaccessible to practitioners.

In response to this gap, this report explores the design and implementation of a minimal yet efficient SAT solver, inspired by MINISAT. We focus on providing an accessible path from theoretical concepts to practical code, allowing readers to understand, extend, and apply SAT techniques in various domains. By presenting the structure and methods of a conflict-driven SAT solver, we offer insight into the critical mechanisms that enable fast and flexible problem-solving—unit propagation, clause learning, and constraint management. This report also presents timing results and discusses applications, providing both a theoretical and practical foundation for further exploration of SAT solvers in advanced computational settings.

Ever since 2003, MiniSat has been a helpful hand to the SAT community, providing them with a small, efficient, and readable SAT solver. 

\subsection{Motivation}
Please note that the first paragraph of a section or subsection is
not indented. The first paragraph that follows a table, figure,
equation etc. does not need an indent, either.

Subsequent paragraphs, however, are indented.

\subsubsection{Sample Heading (Third Level)} Only two levels of
headings should be numbered. Lower level headings remain unnumbered;
they are formatted as run-in headings.

\paragraph{Sample Heading (Fourth Level)}
The contribution should contain no more than four levels of
headings. Table~\ref{tab1} gives a summary of all heading levels.

\begin{table}
\caption{Table captions should be placed above the
tables.}\label{tab1}
\begin{tabular}{|l|l|l|}
\hline
Heading level &  Example & Font size and style\\
\hline
Title (centered) &  {\Large\bfseries Lecture Notes} & 14 point, bold\\
1st-level heading &  {\large\bfseries 1 Introduction} & 12 point, bold\\
2nd-level heading & {\bfseries 2.1 Printing Area} & 10 point, bold\\
3rd-level heading & {\bfseries Run-in Heading in Bold.} Text follows & 10 point, bold\\
4th-level heading & {\itshape Lowest Level Heading.} Text follows & 10 point, italic\\
\hline
\end{tabular}
\end{table}


\noindent Displayed equations are centered and set on a separate
line.
\begin{equation}
x + y = z
\end{equation}
Please try to avoid rasterized images for line-art diagrams and
schemas. Whenever possible, use vector graphics instead (see
Fig.~\ref{fig1}).

\begin{figure}
\includegraphics[width=\textwidth]{fig1.eps}
\caption{A figure caption is always placed below the illustration.
Please note that short captions are centered, while long ones are
justified by the macro package automatically.} \label{fig1}
\end{figure}

\begin{theorem}
This is a sample theorem. The run-in heading is set in bold, while
the following text appears in italics. Definitions, lemmas,
propositions, and corollaries are styled the same way.
\end{theorem}
%
% the environments 'definition', 'lemma', 'proposition', 'corollary',
% 'remark', and 'example' are defined in the LLNCS documentclass as well.
%
\begin{proof}
Proofs, examples, and remarks have the initial word in italics,
while the following text appears in normal font.
\end{proof}
For citations of references, we prefer the use of square brackets
and consecutive numbers. Citations using labels or the author/year
convention are also acceptable. The following bibliography provides
a sample reference list with entries for journal
articles~\cite{ref_article1}, an LNCS chapter~\cite{ref_lncs1}, a
book~\cite{ref_book1}, proceedings without editors~\cite{ref_proc1},
and a homepage~\cite{ref_url1}. Multiple citations are grouped
\cite{ref_article1,ref_lncs1,ref_book1},
\cite{ref_article1,ref_book1,ref_proc1,ref_url1}.

\begin{credits}
\subsubsection{\ackname} A bold run-in heading in small font size at the end of the paper is
used for general acknowledgments, for example This study was funded
by X (grant number Y).

\subsubsection{\discintname}
It is now necessary to declare any competing interests or to specifically
state that the authors have no competing interests. Please place the
statement with a bold run-in heading in small font size beneath the
(optional) acknowledgments\footnote{If EquinOCS, our proceedings submission
system, is used, then the disclaimer can be provided directly in the system.},
for example: The authors have no competing interests to declare that are
relevant to the content of this article. Or: Author A has received research
grants from Company W. Author B has received a speaker honorarium from
Company X and owns stock in Company Y. Author C is a member of committee Z.
\end{credits}
%
% ---- Bibliography ----
%
% BibTeX users should specify bibliography style 'splncs04'.
% References will then be sorted and formatted in the correct style.
%
% \bibliographystyle{splncs04}
% \bibliography{mybibliography}
%
\newpage
{\noindent \huge \textbf{List of acronyms}\par}

\begin{acronym}
 \acro{SAT}{Boolean satisfiability problem}
\end{acronym}
\begin{thebibliography}{8}
\bibitem{ref_article1}
Authors: Niklas Een, Niklas Sorensson; Institute: Chalmers University of Technology, Sweden; Paper title: An Extensible SAT-solver[extended version 1.2]

\end{thebibliography}
\end{document}